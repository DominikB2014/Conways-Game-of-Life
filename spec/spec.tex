\documentclass[12pt]{article}

\usepackage{graphicx}
\usepackage{paralist}
\usepackage{amsfonts}
\usepackage{amsmath}
\usepackage{hhline}
\usepackage{booktabs}
\usepackage{multirow}
\usepackage{multicol}
\usepackage{url}

\oddsidemargin -10mm
\evensidemargin -10mm
\textwidth 160mm
\textheight 200mm
\renewcommand\baselinestretch{1.0}

\pagestyle {plain}
\pagenumbering{arabic}

\newcounter{stepnum}

%% Comments

\usepackage{color}

\title{Assignment 4, Specification}
\author{Dominik Buszowiecki}

\begin {document}

\maketitle

\newpage

  %%%%%%%%%%%%%%%%  Board Module %%%%%%%%%%%%%%%%
\section* {Game Board ADT Module}

\subsection*{Template Module}

BoardT

\subsection* {Uses}

\noindent None

\subsection* {Syntax}

\subsubsection* {Exported Access Programs}

\begin{tabular}{| l | l | l | l |}
\hline
\textbf{Routine name} & \textbf{In} & \textbf{Out} & \textbf{Exceptions}\\
\hline
new BoardT  & Grid & BoardT & invalid\_argument\\
\hline
nextStage & & BoardT & \\
\hline
  toGrid & & Grid & \\
\hline
\end{tabular}

\subsection* {Semantics}

\subsubsection* {State Variables}

$S$ : Grid \textit{\# 2D array}

\subsubsection* {State Invariant}

$\forall e \in S (|S[0]| = |e|)$ \textit{\# All elements in S are sequences of the same size}

\subsubsection* {Assumptions \& Design Decisions}

\begin{itemize}

\item The BoardT constructor is called before any other access
  routine is called on that instance. Once a BoardT has been created, the
  constructor will not be called on it again.

\end{itemize}

\newpage

  %%%%%%%%%%%%%%%%%%%%%%%%%%%%%%%%%%%%

\subsubsection* {Access Routine Semantics}

\noindent BoardT($\mathit{G}$):
\begin{itemize}
\item transition: $S := G$
\item output: $out := self$
\item exception: $exc := (|G| = 0 \lor \lnot(\forall e \in G : |G[0]| = |e|) \Rightarrow \text{invalid\_argument})$
  \\ \textit{\# Exception checks state invariant}
\end{itemize}

\noindent nextStage():
\begin{itemize}
\item transition:
$S:= G$ such that $(\forall i, j: \mathbb{N} | \text{IsInBounds}(S, i, j) : G[i][j] = \text{updateCell}(S, i, j))$
\item output: $out := self$
\item exception: none
\end{itemize}

\noindent toGrid():
\begin{itemize}
\item output: $out := S$
\item exception: none
\end{itemize}

\subsection*{Local Types}

Grid = seq of (seq of $\mathbb{B}$)

\newpage

  %%%%%%%%%%%%%%%%%%%%%%%%%%%%%%%%%%%%

\subsection*{Local Functions}

\noindent $\text{UpdateCell} : \text{Grid} \times \mathbb{N} \times \mathbb{N} \rightarrow \ \mathbb{B} $\\
\noindent
$\text{UpdateCell}(S,x,y) \equiv$

\begin{tabular}{|p{3cm}|p{4cm}|2|}
\hhline{|-|-|-|}
$S[x][y] = \text{True}$  & $\text{NumAdj}(S,x,y) < 2$ & False\\
\hhline{|~|-|-|}
 & $\text{NumAdj}(S,x,y) = 2$ & True\\
\hhline{|~|-|-|}
 & $\text{NumAdj}(S,x,y) = 3$ & True\\
\hhline{|~|-|-|}
 & $\text{NumAdj}(S,x,y) > 3$ & False\\
\hhline{|-|-|-|}
$S[x][y] = \text{False}$ & & $\text{NumAdj}(S,x,y) = 3$\\
\hhline{|-|-|-|}
\end{tabular}\\\\

\noindent $\text{NumAdj} : \text{Grid} \times \mathbb{N} \times \mathbb{N} \rightarrow \ \mathbb{N} $\\
\noindent
$\text{NumAdj}(S,x,y) \equiv$ \\
$ +(i,j: \mathbb{N} | i \in \{x+1, x-1\} \land j \in \{y+1, y-1\} \land \text{IsInBounds}(S,i,j) \land S[i][j] = \text{True} : 1)$ \\
\textit{\# Assumed that when x or y is 0, x-1 or y-1 does not return anything, i or j can only become \{x+1\} or \{y+1\}}\\

\noindent $\text{IsInBounds} : \text{Grid} \times \mathbb{N} \times \mathbb{N} \rightarrow \ \mathbb{B} $\\
\noindent
$\text{IsInBounds}(S,x,y) \equiv x < |S| \land y < |S[0]|$

\newpage

  %%%%%%%%%%%%%%%%  View Module %%%%%%%%%%%%%%%%
\section* {View Module}

\subsection*{Module}

View

\subsection* {Uses}

\noindent GameBoardT

\subsection* {Syntax}

\subsubsection* {Exported Access Programs}

\begin{tabular}{| l | l | l | l |}
\hline
\textbf{Routine name} & \textbf{In} & \textbf{Out} & \textbf{Exceptions}\\
\hline
readStage  & $s:$ string & & file\_not\_found\\
\hline
initStage  & $\mathbb{N}, \mathbb{N}$ & & out\_of\_range\\
\hline
simulate & $\mathbb{N}$ & & \\
\hline
writeStage & $s:$ string & & \\
\hline
\end{tabular}

\subsection* {Semantics}

\subsubsection* {State Variables}

$gameBoard$ : GameBoardT \textit{\# 2D array}

\subsection* {Environment Variables}

gridFile: File containing a grid representation of the game

\subsubsection* {State Invariant}

None

\subsubsection* {Assumptions \& Design Decisions}

\begin{itemize}
    \item The input file will match the given specification, each line will have equal number of characters
    \item The user will either call readStage or initStage before simulating or witting the board
\end{itemize}

\newpage

  %%%%%%%%%%%%%%%%%%%%%%%%%%%%%%%%%%%%

\subsubsection* {Access Routine Semantics}

\noindent readStage($s$):
\begin{itemize}
\item transition: $gameBoard := \text{GameBoard}(G)$ where G is a seq of(seq of $\mathbb{B}$)

    G is generated from the file gridFile associated with the string s.
    It is generated with the following condition:
    \begin{equation*}
        (\forall i: \mathbb{N} | i < |L| : G[i] = \text{stringToRow}(L))
    \end{equation*}
    Where $L$ is a seq of string, each string in L corresponds to a line in file s.
    Therefore, $L[2][4]$ would represent the 5th character in the 3rd row.
    An example of file s is provided below:
    \begin{center}
        \_\_\_\_\_\_\_0\_\_\_\_\_\_\_0\_\_ \\
        \_\_\_0\_\_\_\_\_\_\_\_\_0\_\_\_\_ \\
        \_\_\_\_\_0\_\_\_0\_\_\_\_\_\_\_\_ \\
    \end{center}
    Each "0" corresponds to a populated cell, where a "\_" is an empty cell
    (note: an empty cell can be represented by anything but 0 when the file is read)
\end{itemize}

\noindent initStage(x, y):
\begin{itemize}
\item transition: $gameBoard := G$ such that \\
    $(|G.\text{toGrid}()| = x \land |G.\text{toGrid}()[0]| = y \land (\forall i, j : \mathbb{N}| i < x \land j < y : G.\text{toGrid}()[i][j] = False)$
\item exception: $exc := (x \le 0 \lor y \le 0 \Rightarrow \text{invalid\_argument})$
\end{itemize}

\noindent simulate(n):
\begin{itemize}
\item transition: \textit{\# Procedural Specification} \\
printStage() \\
for all $i$ in $[0..n] : $\\
\hspace{1}$gameBoard = gameBoard.\text{nextStage}()$ \\
\hspace{1}printStage()
\item exception: $exc := (x \le 0 \lor y \le 0 \Rightarrow \text{invalid\_argument})$
\end{itemize}

\noindent writeStage($s$):
\begin{itemize}
\item transition:
    Writes G into the file with name s.
    G can be represented be the following.
    \begin{equation*}
        (\forall i, j: \mathbb{N} | i < |L| \land j < |L[0]| : L[i][j] = "\_" \Rightarrow G[i][j] = \text{False} | L[i][j] = "0" \Rightarrow G[i][j] = \text{True})
    \end{equation*}
    Where $L$ is a seq of string, each string in L corresponds to a line in file s.
\end{itemize}

    %%%%%%%%%%%%%%%%%%%%%%%%%%%%%%%%%%%%

\subsection*{Local Functions}

\noindent $\text{stringToRow} : \text{string} \rightarrow \ \text{seq of }\mathbb{B} $\\
\noindent
$\text{stringToRow}(s) \equiv L \text{ such that } (\forall i : \mathbb{N} | i < |s| : L[i] = (S[i] = "0"))$ \\

\noindent $\text{printStage} :$\\
\noindent
$\text{printStage}() \equiv$ \\
Displays $s$ in terminal where $s$ is: \\
s = $(+i: \mathbb{N} | i < |gameBoard.\text{toGrid}()| : \text{rowToString}(gameBoard.\text{toGrid()}[i]) + "\symbol{92} n")$ \\

\noindent $\text{rowToString} : \text{seq of }\mathbb{B} \rightarrow \text{string}$\\
\noindent
$\text{stringToRow}(s) \equiv L \text{ such that } (\forall i : \mathbb{N} | i < |s| : s[i] \Rightarrow L[i] = "[\#]" \land \lnot s[i] \Rightarrow L[i] = "[\ ]")$ \\

\newpage

  %%%%%%%%%%%%%%%%%%%%%%%%%%%%%%%%%%%%

\section*{Critique of Design}

\wss{Write a critique of the interface for the modules in this project.  Is there
anything missing?  Is there anything you would consider changing?  Why?}\\

\noindent Potential discussion points:

\begin{itemize}
\item The stack module provides a toSeq module that violates essentiality.  To
  address this, another module could be built to provide the toSeq service
  through a function that takes a stack as input and return a sequence.
\end{itemize}

\end {document}
